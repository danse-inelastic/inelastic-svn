\documentclass{article}

% Bitte hier die von Ihnen ben�tigten Style-Files laden
% Please load here the style files you need.
%\usepackage{german}
%\usepackage{amsmath}

\usepackage{oljour}

\begin{document}

% Bitte entfernen Sie das Kommentarzeichen vor dem K�rzel der
% Zeitschrift, in der Ihr Beitrag erscheinen soll.
% Please remove the comment char in the line that contains
% the shortcut for the journal where your submission shall
% appear

%\journalname{an}  % (analysis)
%\journalname{at}  % (Automatisierungstechnik)
%\journalname{it}  % (Information Technology)
%\journalname{ic}  % (icom)
%\journalname{rc}  % (Radiochimica Acta)
%\journalname{sd}  % (Statistics & Decisions)
%\journalname{tm}  % (Technisches Messen)
%\journalname{zk}  % (Zeitschrift f�r Kristallographie)
%\journalname{zp}  % (ZPC)

% Titel des Beitrags (auf deutsch)
% Title of the submission (in german)
\title[de]{}

% Englischer Titel
% Title in english
\title[en]{Molecular Crystal Global Phase Diagrams II: Prototypical crystal packings for group T$_d$ molecules}

% Detaillierte Angaben zu allen Autoren des Beitrags:
% Bitte verwenden Sie f�r jeden Autor eine separate
% (begin/end) author-Umgebung.
% Detailed information about all authors of the submission:
% Please use for every author a separate
% (begin/end) author environment
\begin{author}
  \anumber{}   %fortlaufende Nummer (running number)
  \atitle{}    %akademischer Titel (academic title)
  \firstname{J. Brandon}  %Vorname(n) (first name(s))
  \surname{Keith}    %Nachname (surname)
  \vita{}      %Kurzlebenslauf (short curriculum vitae)
  \institute{Department of Chemical Engineering and Materials
Science, University of Minnesota, Minneapolis, Minnesota}   %Institutsname (name of institute)
  \street{Washington Avenue SE}    %Stra�e (street)
  \number{421}    %Hausnummer (number)
  \zip{55455}      %Postleitzahl (zip code)
  \country{Minneapolis MN, USA}   %Land (country)
  \tel{}      %Telefon (+COUNTRY-AREA-NUMBER: +49-341-49122701)
  \fax{}      %Fax
  \email{}     %E-Mail-Adresse (email address)
\end{author}

\begin{author}
  \anumber{}   %fortlaufende Nummer (running number)
  \atitle{}    %akademischer Titel (academic title)
  \firstname{Richard B.}  %Vorname(n) (first name(s))
  \surname{McClurg}    %Nachname (surname)
  \vita{}      %Kurzlebenslauf (short curriculum vitae)
  \institute{Department of Chemical Engineering and Materials
Science, University of Minnesota, Minneapolis, Minnesota}   %Institutsname (name of institute)
  \street{Washington Avenue SE}    %Stra�e (street)
  \number{421}    %Hausnummer (number)
  \zip{55455}      %Postleitzahl (zip code)
  \country{Minneapolis MN, USA}   %Land (country)
  \tel{}      %Telefon (+COUNTRY-AREA-NUMBER: +49-341-49122701)
  \fax{}      %Fax
  \email{}     %E-Mail-Adresse (email address)
\end{author}

% E-Mail-Adresse f�r die Korrespondenz des Verlags zu diesem
% Beitrag
% Corresponding address (for the publisher) for this submission
\corresponding{jbrkeith@gmail.com}

% englische Zusammenfassung des Beitrags
% english summary of the submission
\abstract{Previously~\cite{Keith04c,Mettes04}, we developed a method
for constructing global phase diagrams (GPDs) for molecular crystals
in which crystal structure is presented as a function of
intermolecular potential parameters. These diagrams are useful for
crystal design and reverse engineering the intermolecular potential
of an experimentally observed crystal structure. We apply this
method to fully ordered single component crystals composed of
tetrahedral molecules in the Cambridge Structural Database. In this
paper we find the number of molecular packings, called reference
lattices, to use in GPDs. We find the structures of these molecules
pack in 15 reference lattices: bcc (25.7\%), fcc (24.3\%), A5
(8.6\%), hcp (8.6\%), sc (8.6\%), and others (24.2\%).}

% deutsche Zusammenfassung des Beitrags
% german summary of the submission
\zusammenfassung{}

% bis zu sechs Keywords (englisch)
% up to six keywords (english)
\keywords{}

% bis zu sechs Schlagw�rter (deutsch)
% up to six keywords (german)
\schlagwort{}

% Widmung
% dedication
\dedication{}

% Die nachfolgenden Angaben werden in der Regel vom Verlag eingetragen.
% The following information is usually supplied by the publisher.
%======================================================================
\received{}
\accepted{}
\volume{}
\issue{}
\class{}
\Year{}
%======================================================================

\maketitle

% Hier bitte den Inhalt des Artikels einf�gen.
% Please insert the content of your submission here.

\end{document}
